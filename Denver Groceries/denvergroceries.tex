\documentclass[]{tufte-handout}

% ams
\usepackage{amssymb,amsmath}

\usepackage{ifxetex,ifluatex}
\usepackage{fixltx2e} % provides \textsubscript
\ifnum 0\ifxetex 1\fi\ifluatex 1\fi=0 % if pdftex
  \usepackage[T1]{fontenc}
  \usepackage[utf8]{inputenc}
\else % if luatex or xelatex
  \makeatletter
  \@ifpackageloaded{fontspec}{}{\usepackage{fontspec}}
  \makeatother
  \defaultfontfeatures{Ligatures=TeX,Scale=MatchLowercase}
  \makeatletter
  \@ifpackageloaded{soul}{
     \renewcommand\allcapsspacing[1]{{\addfontfeature{LetterSpace=15}#1}}
     \renewcommand\smallcapsspacing[1]{{\addfontfeature{LetterSpace=10}#1}}
   }{}
  \makeatother

\fi

% graphix
\usepackage{graphicx}
\setkeys{Gin}{width=\linewidth,totalheight=\textheight,keepaspectratio}

% booktabs
\usepackage{booktabs}

% url
\usepackage{url}

% hyperref
\usepackage{hyperref}

% units.
\usepackage{units}


\setcounter{secnumdepth}{-1}

% citations


% pandoc syntax highlighting
\usepackage{color}
\usepackage{fancyvrb}
\newcommand{\VerbBar}{|}
\newcommand{\VERB}{\Verb[commandchars=\\\{\}]}
\DefineVerbatimEnvironment{Highlighting}{Verbatim}{commandchars=\\\{\}}
% Add ',fontsize=\small' for more characters per line
\newenvironment{Shaded}{}{}
\newcommand{\AlertTok}[1]{\textcolor[rgb]{1.00,0.00,0.00}{\textbf{#1}}}
\newcommand{\AnnotationTok}[1]{\textcolor[rgb]{0.38,0.63,0.69}{\textbf{\textit{#1}}}}
\newcommand{\AttributeTok}[1]{\textcolor[rgb]{0.49,0.56,0.16}{#1}}
\newcommand{\BaseNTok}[1]{\textcolor[rgb]{0.25,0.63,0.44}{#1}}
\newcommand{\BuiltInTok}[1]{#1}
\newcommand{\CharTok}[1]{\textcolor[rgb]{0.25,0.44,0.63}{#1}}
\newcommand{\CommentTok}[1]{\textcolor[rgb]{0.38,0.63,0.69}{\textit{#1}}}
\newcommand{\CommentVarTok}[1]{\textcolor[rgb]{0.38,0.63,0.69}{\textbf{\textit{#1}}}}
\newcommand{\ConstantTok}[1]{\textcolor[rgb]{0.53,0.00,0.00}{#1}}
\newcommand{\ControlFlowTok}[1]{\textcolor[rgb]{0.00,0.44,0.13}{\textbf{#1}}}
\newcommand{\DataTypeTok}[1]{\textcolor[rgb]{0.56,0.13,0.00}{#1}}
\newcommand{\DecValTok}[1]{\textcolor[rgb]{0.25,0.63,0.44}{#1}}
\newcommand{\DocumentationTok}[1]{\textcolor[rgb]{0.73,0.13,0.13}{\textit{#1}}}
\newcommand{\ErrorTok}[1]{\textcolor[rgb]{1.00,0.00,0.00}{\textbf{#1}}}
\newcommand{\ExtensionTok}[1]{#1}
\newcommand{\FloatTok}[1]{\textcolor[rgb]{0.25,0.63,0.44}{#1}}
\newcommand{\FunctionTok}[1]{\textcolor[rgb]{0.02,0.16,0.49}{#1}}
\newcommand{\ImportTok}[1]{#1}
\newcommand{\InformationTok}[1]{\textcolor[rgb]{0.38,0.63,0.69}{\textbf{\textit{#1}}}}
\newcommand{\KeywordTok}[1]{\textcolor[rgb]{0.00,0.44,0.13}{\textbf{#1}}}
\newcommand{\NormalTok}[1]{#1}
\newcommand{\OperatorTok}[1]{\textcolor[rgb]{0.40,0.40,0.40}{#1}}
\newcommand{\OtherTok}[1]{\textcolor[rgb]{0.00,0.44,0.13}{#1}}
\newcommand{\PreprocessorTok}[1]{\textcolor[rgb]{0.74,0.48,0.00}{#1}}
\newcommand{\RegionMarkerTok}[1]{#1}
\newcommand{\SpecialCharTok}[1]{\textcolor[rgb]{0.25,0.44,0.63}{#1}}
\newcommand{\SpecialStringTok}[1]{\textcolor[rgb]{0.73,0.40,0.53}{#1}}
\newcommand{\StringTok}[1]{\textcolor[rgb]{0.25,0.44,0.63}{#1}}
\newcommand{\VariableTok}[1]{\textcolor[rgb]{0.10,0.09,0.49}{#1}}
\newcommand{\VerbatimStringTok}[1]{\textcolor[rgb]{0.25,0.44,0.63}{#1}}
\newcommand{\WarningTok}[1]{\textcolor[rgb]{0.38,0.63,0.69}{\textbf{\textit{#1}}}}

% table with pandoc

% multiplecol
\usepackage{multicol}

% strikeout
\usepackage[normalem]{ulem}

% morefloats
\usepackage{morefloats}


% tightlist macro required by pandoc >= 1.14
\providecommand{\tightlist}{%
  \setlength{\itemsep}{0pt}\setlength{\parskip}{0pt}}

% title / author / date
\title{Which Grocery Store\ldots Denver?}
\author{Andrew McCartney}
\date{1/18/2022}

\usepackage{amsmath}
\usepackage{booktabs}
\usepackage{caption}
\usepackage{longtable}

\begin{document}

\maketitle




\hypertarget{motivation}{%
\subsection{Motivation}\label{motivation}}

Imagine: I'm in the group chat. My friend Arielle is psyching herself up
to go get groceries, and upon a query, writes, ``it's the worst. in part
because my grocery store is the worst. I think it would be less
stressful if I had a better store. it's understocked, so poorly staffed,
checkout takes ages and either you have to bag your own groceries or you
have to wait forever in line just to do self checkout but they still
don't turn on more of the self check out kiosks for some reason. outside
lots of unsavory types loiter and it's uncomfortable, everything smells
like piss.''

Then Laura writes: ``Andrew you should map all the grocery stores by
driving distance and reviews to determine the best option.''

\textbf{Say no more, fam}.

\hypertarget{initial-code}{%
\subsection{Initial Code}\label{initial-code}}

So to get started, we need to think about how we are going to access
data on all the grocery stores nearby my friend's house. I've previously
done an analysis like this for the groupchat when Arielle's sister was
moving to Salt Lake City and needed to find housing. There turned out to
be a Craigslist API then. And the time I needed to pull song lyrics for
a witter bot: Genius.com has an API too. Surely google maps will let me
access their reviews for all the nearby stores?

I do a little googling and I find that there's a new app--new to me at
least--called \texttt{googleway}. It covers a host of map related APIs
that I didn't realize existed. Do I need to generate an API key? I go to
their website and I'm stunned to find I have a google maps API key from
another project from 2.5 years ago live and ready to go.

Initial fumbling with this gives me some locked results and I realize
that for each individual service that the googleway / google maps API
allows, I need to go to my API key and enable that service.

The first two commands are for the \texttt{googleway} API. My key is
saved in a not-included setup chunk for privacy. The third line here is
registering with the \texttt{ggmaps} package--my old standby.

(I once made a shiny app that called down the google maps API any time a
user toggled anything. I was running several hundred requests a session
just to look at different maps of Africa. I'm reasonably certain
\textbf{I'm personally responsible} for API requireing an account and a
key now)

\begin{Shaded}
\begin{Highlighting}[]
\FunctionTok{set\_key}\NormalTok{(}\AttributeTok{key =}\NormalTok{ key)}
\FunctionTok{google\_keys}\NormalTok{()}
\FunctionTok{register\_google}\NormalTok{(}\AttributeTok{key =}\NormalTok{ key)}
\end{Highlighting}
\end{Shaded}

Having logged in and connected, we need to run our first data pull.

\begin{Shaded}
\begin{Highlighting}[]
\NormalTok{res }\OtherTok{\textless{}{-}} \FunctionTok{google\_places}\NormalTok{(}\AttributeTok{location =}\NormalTok{ arielle,}
                     \AttributeTok{keyword =} \StringTok{"Grocery Store"}\NormalTok{,}
                     \AttributeTok{radius =} \DecValTok{5000}\NormalTok{,}
                     \AttributeTok{key =}\NormalTok{ key)}
\end{Highlighting}
\end{Shaded}

this is calling the \texttt{arielle} object I made earlier, which is a
list of two elements representing her longitude and latitude in Denver.

I expected the returned object from this to be a data frame, so you can
imagine my befuddlement the first few times I ran this and got a url in
the console. The first few URLs I plugged in threw the error I mentioned
above (access denied) and then I was able to run it properly. The next
code calls on that URL and converts its content from JSON format to a
tidy data frame that I can use.

So in this place, which I cannot show because it would expose my key, I
have code to the effect of:

\begin{verbatim}
groceries <-jsonlite::fromJSON("https://url generated by last step goes here")
\end{verbatim}

and I use the results of that to generate a tibble calling on the
following nested lists:

\begin{Shaded}
\begin{Highlighting}[]
\NormalTok{groceries\_tib}\OtherTok{\textless{}{-}}\FunctionTok{cbind}\NormalTok{(groceries}\SpecialCharTok{$}\NormalTok{result}\SpecialCharTok{$}\NormalTok{name,}
\NormalTok{      groceries}\SpecialCharTok{$}\NormalTok{result}\SpecialCharTok{$}\NormalTok{geometry}\SpecialCharTok{$}\NormalTok{location}\SpecialCharTok{$}\NormalTok{lat,}
\NormalTok{      groceries}\SpecialCharTok{$}\NormalTok{result}\SpecialCharTok{$}\NormalTok{geometry}\SpecialCharTok{$}\NormalTok{location}\SpecialCharTok{$}\NormalTok{lng,}
\NormalTok{      groceries}\SpecialCharTok{$}\NormalTok{result}\SpecialCharTok{$}\NormalTok{rating, }
\NormalTok{      groceries}\SpecialCharTok{$}\NormalTok{result}\SpecialCharTok{$}\NormalTok{user\_ratings\_total) }\SpecialCharTok{\%\textgreater{}\%} 
  \FunctionTok{as\_tibble}\NormalTok{()}
\NormalTok{gro1}\OtherTok{\textless{}{-}}\NormalTok{groceries\_tib }\SpecialCharTok{\%\textgreater{}\%} 
  \FunctionTok{rename}\NormalTok{(}\AttributeTok{name=}\NormalTok{V1, }
         \AttributeTok{latitude =}\NormalTok{ V2,}
         \AttributeTok{longitude =}\NormalTok{ V3, }
         \AttributeTok{rating =}\NormalTok{ V4, }
         \AttributeTok{n\_ratings =}\NormalTok{ V5) }\SpecialCharTok{\%\textgreater{}\%} 
  \FunctionTok{arrange}\NormalTok{(}\FunctionTok{desc}\NormalTok{(rating))}
\end{Highlighting}
\end{Shaded}

But because there is a rate limit of 20 and I had already decided to
pull the first 40 grocery stores, I have to run this twice to get
\texttt{gro2} to match \texttt{gro1} that I just created. The second run
is a bit more intensive because we are calling a ``next page'' of
results from google.

\begin{Shaded}
\begin{Highlighting}[]
\NormalTok{res\_next }\OtherTok{\textless{}{-}} \FunctionTok{google\_places}\NormalTok{(}\AttributeTok{location =}\NormalTok{ arielle,}
                          \AttributeTok{keyword =} \StringTok{"Grocery Store"}\NormalTok{,}
                          \AttributeTok{page\_token =}\NormalTok{ groceries}\SpecialCharTok{$}\NormalTok{next\_page\_token,}
                          \AttributeTok{radius =} \DecValTok{5000}\NormalTok{,}
                          \AttributeTok{key =}\NormalTok{ key)}
\end{Highlighting}
\end{Shaded}

This should be the last code I have to hide for a while, but I did the
same URL -\textgreater{} JSON business I did with the first section. I
clean it up:

\begin{Shaded}
\begin{Highlighting}[]
\NormalTok{groceries\_2\_tib}\OtherTok{\textless{}{-}}\FunctionTok{cbind}\NormalTok{(groceries\_2}\SpecialCharTok{$}\NormalTok{result}\SpecialCharTok{$}\NormalTok{name,}
\NormalTok{                   groceries\_2}\SpecialCharTok{$}\NormalTok{result}\SpecialCharTok{$}\NormalTok{geometry}\SpecialCharTok{$}\NormalTok{location}\SpecialCharTok{$}\NormalTok{lat,}
\NormalTok{                   groceries\_2}\SpecialCharTok{$}\NormalTok{result}\SpecialCharTok{$}\NormalTok{geometry}\SpecialCharTok{$}\NormalTok{location}\SpecialCharTok{$}\NormalTok{lng,}
\NormalTok{                   groceries\_2}\SpecialCharTok{$}\NormalTok{result}\SpecialCharTok{$}\NormalTok{rating, }
\NormalTok{                   groceries\_2}\SpecialCharTok{$}\NormalTok{result}\SpecialCharTok{$}\NormalTok{user\_ratings\_total) }\SpecialCharTok{\%\textgreater{}\%} 
  \FunctionTok{as\_tibble}\NormalTok{()}


\NormalTok{gro2}\OtherTok{\textless{}{-}}\NormalTok{groceries\_2\_tib }\SpecialCharTok{\%\textgreater{}\%} 
  \FunctionTok{rename}\NormalTok{(}\AttributeTok{name=}\NormalTok{V1, }
         \AttributeTok{latitude =}\NormalTok{ V2,}
         \AttributeTok{longitude =}\NormalTok{ V3, }
         \AttributeTok{rating =}\NormalTok{ V4, }
         \AttributeTok{n\_ratings =}\NormalTok{ V5) }\SpecialCharTok{\%\textgreater{}\%} 
  \FunctionTok{arrange}\NormalTok{(}\FunctionTok{desc}\NormalTok{(rating))}
\end{Highlighting}
\end{Shaded}

And now we can join them together into one proper data frame and examine
what we have.

\begin{Shaded}
\begin{Highlighting}[]
\NormalTok{gro}\OtherTok{\textless{}{-}}\NormalTok{dplyr}\SpecialCharTok{::}\FunctionTok{bind\_rows}\NormalTok{(gro1,gro2)}


\NormalTok{gro }\SpecialCharTok{\%\textgreater{}\%} 
  \FunctionTok{arrange}\NormalTok{(}\FunctionTok{desc}\NormalTok{(rating)) }\SpecialCharTok{\%\textgreater{}\%} 
  \FunctionTok{head}\NormalTok{(}\DecValTok{10}\NormalTok{) }\SpecialCharTok{\%\textgreater{}\%} 
\NormalTok{  gt}\SpecialCharTok{::}\FunctionTok{gt}\NormalTok{() }\SpecialCharTok{\%\textgreater{}\%} 
\NormalTok{   gt}\SpecialCharTok{::}\FunctionTok{tab\_header}\NormalTok{(}
    \AttributeTok{title =} \StringTok{"Top Ten Rated Grocery Stores in Denver"}
\NormalTok{  ) }
\end{Highlighting}
\end{Shaded}

\captionsetup[table]{labelformat=empty,skip=1pt}
\begin{longtable}{lllll}
\caption*{
\large Top Ten Rated Grocery Stores in Denver\\ 
} \\ 
\toprule
name & latitude & longitude & rating & n\_ratings \\ 
\midrule
Ready Foods Inc & 39.7401931 & -105.0208209 & 5 & 1 \\ 
Sun Market & 39.7496388 & -104.9711721 & 4.9 & 21 \\ 
Town \& Country Market Produce & 39.7399005 & -104.9371343 & 4.9 & 17 \\ 
Pete's Fruits \& Vegetables & 39.7129002 & -104.9220065 & 4.7 & 164 \\ 
Denver Market & 39.7715998 & -105.006592 & 4.6 & 27 \\ 
Trader Joe's & 39.7268649 & -104.9827877 & 4.6 & 2295 \\ 
Marczyk Fine Foods & 39.7429609 & -104.9780866 & 4.6 & 771 \\ 
12th Ave Market & 39.7349414 & -104.9558239 & 4.5 & 62 \\ 
Trader Joe's & 39.7287708 & -104.9403211 & 4.5 & 3497 \\ 
Leevers Locavore & 39.7689136 & -105.019854 & 4.5 & 411 \\ 
\bottomrule
\end{longtable}

This tells us some good information: first of all, the highest-rated
grocery store only has one review, so that's dubious and we need to keep
that in mind from now on. It also puts two Trader Joe's in the top 10,
which strikes me as appropriate.

What about the bottom ten?

\begin{Shaded}
\begin{Highlighting}[]
\NormalTok{gro }\SpecialCharTok{\%\textgreater{}\%} 
  \FunctionTok{arrange}\NormalTok{(}\FunctionTok{desc}\NormalTok{(rating)) }\SpecialCharTok{\%\textgreater{}\%} 
  \FunctionTok{tail}\NormalTok{(}\DecValTok{10}\NormalTok{) }\SpecialCharTok{\%\textgreater{}\%} 
\NormalTok{  gt}\SpecialCharTok{::}\FunctionTok{gt}\NormalTok{()}\SpecialCharTok{\%\textgreater{}\%} 
\NormalTok{   gt}\SpecialCharTok{::}\FunctionTok{tab\_header}\NormalTok{(}
    \AttributeTok{title =} \StringTok{"Ten Worst Grocery Stores in Denver"}
\NormalTok{  ) }
\end{Highlighting}
\end{Shaded}

\captionsetup[table]{labelformat=empty,skip=1pt}
\begin{longtable}{lllll}
\caption*{
\large Ten Worst Grocery Stores in Denver\\ 
} \\ 
\toprule
name & latitude & longitude & rating & n\_ratings \\ 
\midrule
King Soopers & 39.7310976 & -104.9735003 & 4 & 1364 \\ 
Choice Market & 39.744453 & -104.987074 & 4 & 197 \\ 
Safeway & 39.755634 & -105.0242664 & 3.9 & 1636 \\ 
Argonant Groceries & 39.7402164 & -104.9838854 & 3.9 & 11 \\ 
King Soopers & 39.7374907 & -104.9176658 & 3.9 & 485 \\ 
King Soopers & 39.737521 & -104.997947 & 3.8 & 1159 \\ 
Town Grocery & 39.7574123 & -104.9736486 & 3.7 & 3 \\ 
Rio Grande Grocery Store & 39.7290492 & -105.0022845 & 3.7 & 7 \\ 
Safeway & 39.7486925 & -104.9775013 & 3.6 & 2352 \\ 
Decatur Fresh & 39.7327007 & -105.0218541 & 0 & 0 \\ 
\bottomrule
\end{longtable}

Ah. So the safeway (known to my friend as the ``unsafeway'') in question
is actually the lowest rated grocery store in Denver (at least first 40
hits) that has more than 1 rating. So we know we're dealing with some
real garbage here.

\hypertarget{distance}{%
\subsection{Distance}\label{distance}}

\hypertarget{haversine}{%
\subsubsection{Haversine}\label{haversine}}

In my original version of this, I just wanted to knock off and go eat
dinner and produce a result ASAP, so I did an old standby and adjusted
the data set to create a haversine distance from each grocery store to
my friend's coordinates. That was simple because I already had code to
that effect sitting around from a project in graduate school.

\begin{Shaded}
\begin{Highlighting}[]
\CommentTok{\# degrees to radians}
\NormalTok{deg2rad}\OtherTok{\textless{}{-}}\ControlFlowTok{function}\NormalTok{(d)\{}
  \FunctionTok{return}\NormalTok{(d }\SpecialCharTok{*}\NormalTok{ (pi}\SpecialCharTok{/}\DecValTok{180}\NormalTok{))}
\NormalTok{\}}

\CommentTok{\# calculate haversine distance between two coordinates. }
\NormalTok{haversine\_km}\OtherTok{\textless{}{-}}\ControlFlowTok{function}\NormalTok{(lat1,lon1,lat2,lon2)\{}
\NormalTok{  Rad }\OtherTok{\textless{}{-}} \DecValTok{6371} \CommentTok{\# Radius of earth in km}
\NormalTok{  dLat }\OtherTok{\textless{}{-}} \FunctionTok{deg2rad}\NormalTok{(lat2}\SpecialCharTok{{-}}\NormalTok{lat1)}
\NormalTok{  dLon }\OtherTok{\textless{}{-}} \FunctionTok{deg2rad}\NormalTok{(lon2}\SpecialCharTok{{-}}\NormalTok{lon1)}
\NormalTok{  a }\OtherTok{\textless{}{-}} \FunctionTok{sin}\NormalTok{(dLat}\SpecialCharTok{/}\DecValTok{2}\NormalTok{)}\SpecialCharTok{*}\FunctionTok{sin}\NormalTok{(dLat}\SpecialCharTok{/}\DecValTok{2}\NormalTok{) }\SpecialCharTok{+} \FunctionTok{cos}\NormalTok{(}\FunctionTok{deg2rad}\NormalTok{(lat1)) }\SpecialCharTok{*} \FunctionTok{cos}\NormalTok{(}\FunctionTok{deg2rad}\NormalTok{(lat2)) }\SpecialCharTok{*} \FunctionTok{sin}\NormalTok{(dLon}\SpecialCharTok{/}\DecValTok{2}\NormalTok{) }\SpecialCharTok{*} \FunctionTok{sin}\NormalTok{(dLon}\SpecialCharTok{/}\DecValTok{2}\NormalTok{)}
\NormalTok{  c }\OtherTok{\textless{}{-}} \DecValTok{2} \SpecialCharTok{*} \FunctionTok{atan2}\NormalTok{(}\FunctionTok{sqrt}\NormalTok{(a), }\FunctionTok{sqrt}\NormalTok{(}\DecValTok{1}\SpecialCharTok{{-}}\NormalTok{a))}
\NormalTok{  d }\OtherTok{\textless{}{-}}\NormalTok{ Rad}\SpecialCharTok{*}\NormalTok{c }\CommentTok{\# distance in km}
  \FunctionTok{return}\NormalTok{(d)}
\NormalTok{\}}
\end{Highlighting}
\end{Shaded}

Then it was a simple matter of mutating a new column with that
function\ldots. except it was at this point that I first noticed the
problem that my variables were all in \texttt{character} and would need
to be changed to \texttt{as.numeric} to do any kind of calculations.
whoops.

\begin{Shaded}
\begin{Highlighting}[]
\NormalTok{gro }\OtherTok{\textless{}{-}}\NormalTok{ gro }\SpecialCharTok{\%\textgreater{}\%} 
  \FunctionTok{mutate}\NormalTok{(}\AttributeTok{latitude =} \FunctionTok{as.numeric}\NormalTok{(latitude),}
         \AttributeTok{longitude=} \FunctionTok{as.numeric}\NormalTok{(longitude),}
         \AttributeTok{rating =} \FunctionTok{as.numeric}\NormalTok{(rating),}
         \AttributeTok{n\_ratings =} \FunctionTok{as.numeric}\NormalTok{(n\_ratings)}
\NormalTok{         )}

\NormalTok{gro }\OtherTok{\textless{}{-}}\NormalTok{ gro}\SpecialCharTok{\%\textgreater{}\%} 
  \FunctionTok{mutate}\NormalTok{(}\AttributeTok{haversine =} \FunctionTok{haversine\_km}\NormalTok{(latitude, longitude, arielle[}\DecValTok{1}\NormalTok{], arielle[}\DecValTok{2}\NormalTok{]))}
\end{Highlighting}
\end{Shaded}

Having my friend's coordinates stored as a variable ended up being way
handier than I would have anticipated and I think there's a lesson
there. We talk about creating variables up-top and never hard coding
numbers when avoidable, but the lesson goes the other way too, in terms
of making a variable to save yourself time copying and pasting numbers
even if they'll never change in the course of your coding.

\hypertarget{google-route-times}{%
\subsubsection{Google Route Times}\label{google-route-times}}

At this point, I had already made a version of Visualization One down
below, but I challenged myself to re-do it with travel time. Everyone
knows that sometimes a closer location takes longer to get to because
highways, traffic, and ephemera like that. This is especially true in
American cities, where large highways can often cut off adjacent
neighborhoods from one another. So let's dig back into that google API
and see what they have for us.

First, we do two data pulls, which are generated \emph{as} you run the
calculations. So these calls are calculating all 20 routes from each of
the original two \texttt{gro1} and \texttt{gro2} objects to the
\texttt{arielle} object and generating a product called \texttt{drive\_}
as the case needs.

\begin{Shaded}
\begin{Highlighting}[]
\CommentTok{\# split in 2 due to rate limit of 25}
\NormalTok{drive1 }\OtherTok{\textless{}{-}} \FunctionTok{google\_distance}\NormalTok{(}\AttributeTok{origins =}\NormalTok{ gro1[}\DecValTok{1}\SpecialCharTok{:}\DecValTok{20}\NormalTok{, }\FunctionTok{c}\NormalTok{(}\StringTok{"latitude"}\NormalTok{, }\StringTok{"longitude"}\NormalTok{)],}
                      \AttributeTok{destinations =}\NormalTok{ arielle,}
                      \AttributeTok{key =}\NormalTok{ key)}
\NormalTok{drive2 }\OtherTok{\textless{}{-}} \FunctionTok{google\_distance}\NormalTok{(}\AttributeTok{origins =}\NormalTok{ gro2[}\DecValTok{1}\SpecialCharTok{:}\DecValTok{20}\NormalTok{, }\FunctionTok{c}\NormalTok{(}\StringTok{"latitude"}\NormalTok{, }\StringTok{"longitude"}\NormalTok{)],}
                          \AttributeTok{destinations =}\NormalTok{ arielle,}
                          \AttributeTok{key =}\NormalTok{ key)}
\end{Highlighting}
\end{Shaded}

Once these are generated, I use the dirty old trick shown here to apply
the nested data frames on to the \texttt{gro} objects from before:

\begin{Shaded}
\begin{Highlighting}[]
\CommentTok{\# apply the nested list to the original datasets from above}
\NormalTok{gro1}\SpecialCharTok{$}\NormalTok{drive}\OtherTok{\textless{}{-}}\NormalTok{drive1}\SpecialCharTok{$}\NormalTok{rows}\SpecialCharTok{$}\NormalTok{elements}
\NormalTok{gro2}\SpecialCharTok{$}\NormalTok{drive}\OtherTok{\textless{}{-}}\NormalTok{drive2}\SpecialCharTok{$}\NormalTok{rows}\SpecialCharTok{$}\NormalTok{elements}
\end{Highlighting}
\end{Shaded}

This updated both data frames so that the final variable was actually a
list that contained multiple dataframes. See:

\begin{Shaded}
\begin{Highlighting}[]
\NormalTok{gro2 }\SpecialCharTok{\%\textgreater{}\%} 
\NormalTok{  dplyr}\SpecialCharTok{::}\FunctionTok{select}\NormalTok{(name, rating, drive)}
\end{Highlighting}
\end{Shaded}

\begin{verbatim}
## # A tibble: 20 x 3
##    name                          rating drive       
##    <chr>                         <chr>  <list>      
##  1 Sun Market                    4.9    <df [1 x 3]>
##  2 Town & Country Market Produce 4.9    <df [1 x 3]>
##  3 Pete's Fruits & Vegetables    4.7    <df [1 x 3]>
##  4 Trader Joe's                  4.6    <df [1 x 3]>
##  5 Marczyk Fine Foods            4.6    <df [1 x 3]>
##  6 Trader Joe's                  4.5    <df [1 x 3]>
##  7 Leevers Locavore              4.5    <df [1 x 3]>
##  8 Natural Grocers               4.4    <df [1 x 3]>
##  9 Park Hill Supermarket         4.4    <df [1 x 3]>
## 10 Max Market Delicatessen       4.4    <df [1 x 3]>
## 11 Holly Food Market             4.3    <df [1 x 3]>
## 12 Villa Park Mini Mart          4.3    <df [1 x 3]>
## 13 Whole Foods Market            4.2    <df [1 x 3]>
## 14 Natural Grocers               4.2    <df [1 x 3]>
## 15 Gem Store                     4.2    <df [1 x 3]>
## 16 Marczyk Fine Foods            4.2    <df [1 x 3]>
## 17 Safeway                       4.1    <df [1 x 3]>
## 18 Choice Market                 4      <df [1 x 3]>
## 19 Rio Grande Grocery Store      3.7    <df [1 x 3]>
## 20 Decatur Fresh                 0      <df [1 x 3]>
\end{verbatim}

Well! That's a problem but it's on we can deal with. We're going to make
the adjustment for each set (\texttt{gro1} and \texttt{gro2}) and then
re-bind them together. Why not bind first and fix second? I don't know,
I was getting a funny error for dataset 2 but not 1 and I wanted to fix
that first and then I never went back to re-factor the code.

\begin{Shaded}
\begin{Highlighting}[]
\CommentTok{\# unnest again, excluding distance here because we want drive time and the name overlap borks the process. }
\CommentTok{\# we can actually get both if we want by also unnesting distance but that\textquotesingle{}s neither here nor there. }
\NormalTok{gro1 }\OtherTok{\textless{}{-}}\NormalTok{gro1 }\SpecialCharTok{\%\textgreater{}\%} 
\NormalTok{  dplyr}\SpecialCharTok{::}\FunctionTok{select}\NormalTok{(name, latitude, longitude, rating, n\_ratings, drive) }\SpecialCharTok{\%\textgreater{}\%} 
  \FunctionTok{unnest\_wider}\NormalTok{(drive) }\SpecialCharTok{\%\textgreater{}\%} 
  \FunctionTok{unnest\_wider}\NormalTok{(duration) }\SpecialCharTok{\%\textgreater{}\%} 
  \FunctionTok{mutate}\NormalTok{(}\AttributeTok{minutes =}\NormalTok{ value}\SpecialCharTok{/}\DecValTok{60}\NormalTok{)}

\CommentTok{\# fix}
\NormalTok{gro1 }\OtherTok{\textless{}{-}}\NormalTok{gro1 }\SpecialCharTok{\%\textgreater{}\%} 
  \FunctionTok{mutate}\NormalTok{(}\AttributeTok{latitude =} \FunctionTok{as.numeric}\NormalTok{(latitude),}
         \AttributeTok{longitude=} \FunctionTok{as.numeric}\NormalTok{(longitude),}
         \AttributeTok{rating =} \FunctionTok{as.numeric}\NormalTok{(rating),}
         \AttributeTok{n\_ratings =} \FunctionTok{as.numeric}\NormalTok{(n\_ratings)}
\NormalTok{  )}
\CommentTok{\# do all in one step. }
\NormalTok{gro2}\OtherTok{\textless{}{-}}\NormalTok{gro2 }\SpecialCharTok{\%\textgreater{}\%} 
\NormalTok{  dplyr}\SpecialCharTok{::}\FunctionTok{select}\NormalTok{(name, latitude, longitude, rating, n\_ratings, drive) }\SpecialCharTok{\%\textgreater{}\%} 
  \FunctionTok{unnest\_wider}\NormalTok{(drive) }\SpecialCharTok{\%\textgreater{}\%} 
  \FunctionTok{unnest\_wider}\NormalTok{(duration) }\SpecialCharTok{\%\textgreater{}\%} 
  \CommentTok{\# assuming value = seconds (it works out to rounding)}
  \FunctionTok{mutate}\NormalTok{(}\AttributeTok{minutes =}\NormalTok{ value}\SpecialCharTok{/}\DecValTok{60}\NormalTok{) }\SpecialCharTok{\%\textgreater{}\%} 
  \FunctionTok{mutate}\NormalTok{(}\AttributeTok{latitude =} \FunctionTok{as.numeric}\NormalTok{(latitude),}
         \AttributeTok{longitude=} \FunctionTok{as.numeric}\NormalTok{(longitude),}
         \AttributeTok{rating =} \FunctionTok{as.numeric}\NormalTok{(rating),}
         \AttributeTok{n\_ratings =} \FunctionTok{as.numeric}\NormalTok{(n\_ratings)}
\NormalTok{  )}

\CommentTok{\# bind and arrange. }
\NormalTok{gro}\OtherTok{\textless{}{-}}\NormalTok{dplyr}\SpecialCharTok{::}\FunctionTok{bind\_rows}\NormalTok{(gro1,gro2)}
\NormalTok{gro}\OtherTok{\textless{}{-}}\NormalTok{gro }\SpecialCharTok{\%\textgreater{}\%} 
  \FunctionTok{arrange}\NormalTok{(}\FunctionTok{desc}\NormalTok{(rating)) }\SpecialCharTok{\%\textgreater{}\%} 
\NormalTok{  dplyr}\SpecialCharTok{::}\FunctionTok{select}\NormalTok{(}\SpecialCharTok{{-}}\NormalTok{status)}

\DocumentationTok{\#\# pro{-}tip: always append dplyr:: when using select because it often conflicts with other packages.}
\end{Highlighting}
\end{Shaded}

A few notes: I immediately assumed that the \texttt{value} variable
under ``duration'' was seconds, and was pleasantly rewarded when, after
dividing that column by sixty to get a \texttt{minutes} column, that
turned out to be true.

Secondly, why did I unnest \texttt{duration} list but not unnest the
\texttt{distance} list? Because I already had distance covered by the
prior analysis and didn't care about google's opinion on that. I care
about how long it takes my friend to drive there at this point.

On to analysis!

\hypertarget{visualization-one}{%
\subsection{Visualization One}\label{visualization-one}}

First, I generate a second data frame representing the top ten grocery
stores. I'm going to label them on the graph but the labels for the
crappy ones are irrelevant (barring one in particular: the unsafeway).

\begin{Shaded}
\begin{Highlighting}[]
\NormalTok{gro\_top\_n }\OtherTok{\textless{}{-}}\NormalTok{ gro }\SpecialCharTok{\%\textgreater{}\%} 
  \FunctionTok{arrange}\NormalTok{(}\FunctionTok{desc}\NormalTok{(rating)) }\SpecialCharTok{\%\textgreater{}\%} 
  \FunctionTok{head}\NormalTok{(}\DecValTok{10}\NormalTok{)}
\end{Highlighting}
\end{Shaded}

Then plug it all together to get the graph:

\begin{Shaded}
\begin{Highlighting}[]
\NormalTok{gro }\SpecialCharTok{\%\textgreater{}\%} 
  \FunctionTok{filter}\NormalTok{(rating }\SpecialCharTok{\textgreater{}} \FloatTok{3.45}\NormalTok{) }\SpecialCharTok{\%\textgreater{}\%} 
  \FunctionTok{ggplot}\NormalTok{(}\FunctionTok{aes}\NormalTok{(}\AttributeTok{x =}\NormalTok{ minutes, }\AttributeTok{y =}\NormalTok{ rating, }\AttributeTok{size =}\NormalTok{ n\_ratings, }\AttributeTok{color =}\NormalTok{ rating)) }\SpecialCharTok{+}
  \FunctionTok{geom\_text\_repel}\NormalTok{(}\AttributeTok{data=}\NormalTok{ gro\_top\_n, }
            \AttributeTok{mapping =} \FunctionTok{aes}\NormalTok{(}\AttributeTok{x =}\NormalTok{ minutes, }\AttributeTok{y =}\NormalTok{ rating, }\AttributeTok{label =}\NormalTok{ name), }
            \AttributeTok{inherit.aes =} \ConstantTok{FALSE}\NormalTok{, }
            \AttributeTok{fill =} \StringTok{"white"}\NormalTok{,}
            \CommentTok{\#family = "Sofia",}
            \AttributeTok{min.segment.length =} \DecValTok{0}\NormalTok{) }\SpecialCharTok{+}
  \FunctionTok{geom\_point}\NormalTok{() }\SpecialCharTok{+}
  \FunctionTok{labs}\NormalTok{(}\AttributeTok{x =} \StringTok{"Minutes Drive Time (Google Maps)"}\NormalTok{,}
       \AttributeTok{y =} \StringTok{"Average Rating"}\NormalTok{, }
       \AttributeTok{title =} \StringTok{"Which Grocery Store Should Arielle Go to?"}\NormalTok{, }
       \AttributeTok{size =} \StringTok{"Number of Ratings"}\NormalTok{) }\SpecialCharTok{+} 
  \FunctionTok{annotate}\NormalTok{(}\StringTok{"text"}\NormalTok{, }\AttributeTok{x =} \DecValTok{5}\NormalTok{, }\AttributeTok{y =} \FloatTok{3.5}\NormalTok{, }\AttributeTok{label =} \StringTok{"Arielle\textquotesingle{}s Terrible Safeway"}\NormalTok{, }
           \CommentTok{\#family = "Sofia"}
\NormalTok{           ) }\SpecialCharTok{+} 
  \FunctionTok{scale\_color\_viridis\_c}\NormalTok{(}\AttributeTok{direction =} \SpecialCharTok{{-}}\DecValTok{1}\NormalTok{) }\SpecialCharTok{+} 
  \FunctionTok{geom\_curve}\NormalTok{(}\FunctionTok{aes}\NormalTok{(}\AttributeTok{x =} \DecValTok{5}\NormalTok{, }\AttributeTok{y =}  \FloatTok{3.5}\NormalTok{, }\AttributeTok{xend =} \FloatTok{1.72}\NormalTok{, }\AttributeTok{yend =}  \FloatTok{3.6}\NormalTok{),}
             \AttributeTok{arrow =} \FunctionTok{arrow}\NormalTok{(), }\AttributeTok{size =} \FloatTok{0.2}\NormalTok{, }\AttributeTok{color =} \StringTok{"black"}\NormalTok{)}\SpecialCharTok{+} 
  \FunctionTok{scale\_y\_continuous}\NormalTok{(}\AttributeTok{limits =} \FunctionTok{c}\NormalTok{(}\FloatTok{3.45}\NormalTok{,}\DecValTok{5}\NormalTok{)) }\SpecialCharTok{+} 
  \FunctionTok{theme\_light}\NormalTok{() }\SpecialCharTok{+} 
  \CommentTok{\#theme(text = ggplot2::element\_text(family = "Sofia")) +}
  \ConstantTok{NULL}
\end{Highlighting}
\end{Shaded}

\includegraphics{denvergroceries_files/figure-latex/vizone-b-1}

Note that I filtered out any grocery stores whose ratings are below the
unsafeway--that's the two that had 3.0 and 0.0 respectively, and I don't
care about them. They were so far outlying that they were making the
rest of the graph look warped.

This was great and I was really satisfied, then my early-career-mentor
chimed in on social media to say, why not make a voronoi diagram?

\hypertarget{visualization-two}{%
\subsection{Visualization Two}\label{visualization-two}}

The first step here is to get a map that covers the area we are
considering. First, I used the regular \texttt{get\_map} function and
plotted all the grocery stores. Then I resized as needed until I felt
like I had a sense of the coordinates I needed and then used those to
generate the \texttt{boundingbox} variable required by the
\texttt{get\_stamenmap} function. Stamen's maps are more artsy
and--importantly--less text-laden than the google map results, which was
important because I was going to be labeling the grocery stores and I
needed something more clean.

\begin{Shaded}
\begin{Highlighting}[]
\NormalTok{boundingbox}\OtherTok{\textless{}{-}}\FunctionTok{c}\NormalTok{(}\AttributeTok{left =} \SpecialCharTok{{-}}\FloatTok{105.025}\NormalTok{, }\AttributeTok{bottom =} \FloatTok{39.71}\NormalTok{, }\AttributeTok{right =}  \SpecialCharTok{{-}}\FloatTok{104.925}\NormalTok{, }\AttributeTok{top =}\FloatTok{39.79}\NormalTok{ )}
\NormalTok{denver3}\OtherTok{\textless{}{-}}\FunctionTok{get\_stamenmap}\NormalTok{(boundingbox, }\AttributeTok{zoom =} \DecValTok{14}\NormalTok{, }\AttributeTok{maptype =} \StringTok{"toner{-}lite"}\NormalTok{)}

\NormalTok{denver3 }\SpecialCharTok{\%\textgreater{}\%} \FunctionTok{ggmap}\NormalTok{()}
\end{Highlighting}
\end{Shaded}

\includegraphics{denvergroceries_files/figure-latex/get_map-1}

Frustratingly, to make the layer for voronoi diagrams go all the way to
the outer edge of the map, you have to define a separate bounding box
for \emph{them}, which, annoyingly, has a different parameter structure
as well--you have to give the coordinates, in order, for each individual
corner point. The bounding box parameters were just the farthest
longitude or latitude in each particular direction--much easiser. But
\texttt{ggvoronoi} doesn't presume our product is a rectangle, so we
have to give each point.

\begin{Shaded}
\begin{Highlighting}[]
\NormalTok{outline.df }\OtherTok{\textless{}{-}} \FunctionTok{data.frame}\NormalTok{(}\AttributeTok{x =} \FunctionTok{c}\NormalTok{(}\SpecialCharTok{{-}}\FloatTok{105.025}\NormalTok{, }\SpecialCharTok{{-}}\FloatTok{104.925}\NormalTok{,}\SpecialCharTok{{-}}\FloatTok{104.925}\NormalTok{, }\SpecialCharTok{{-}}\FloatTok{105.025}\NormalTok{),}
                         \AttributeTok{y =} \FunctionTok{c}\NormalTok{(}\FloatTok{39.71}\NormalTok{, }\FloatTok{39.71}\NormalTok{,}\FloatTok{39.79}\NormalTok{, }\FloatTok{39.79}\NormalTok{))}
\NormalTok{gro\_filtered}\OtherTok{\textless{}{-}}\NormalTok{gro }\SpecialCharTok{\%\textgreater{}\%} 
  \FunctionTok{filter}\NormalTok{(rating }\SpecialCharTok{\textgreater{}} \FloatTok{3.5}\NormalTok{)      }
\FunctionTok{ggmap}\NormalTok{(denver3,}
      \CommentTok{\# base layer allows the same kind of basics that go in the original ggplot call without a map. }
      \AttributeTok{base\_layer =} \FunctionTok{ggplot}\NormalTok{(}\AttributeTok{data=}\NormalTok{gro\_filtered, }\FunctionTok{aes}\NormalTok{(}\AttributeTok{x =}\NormalTok{ longitude,}\AttributeTok{y=}\NormalTok{latitude, }\AttributeTok{label =}\NormalTok{ name)) }
\NormalTok{      ) }\SpecialCharTok{+}
  \FunctionTok{geom\_point}\NormalTok{()}\SpecialCharTok{+}
\NormalTok{  ggvoronoi}\SpecialCharTok{::}\FunctionTok{geom\_voronoi}\NormalTok{(}
    \FunctionTok{aes}\NormalTok{(}\AttributeTok{fill =}\NormalTok{ rating),}
    \AttributeTok{color =} \StringTok{"black"}\NormalTok{, }
    \AttributeTok{alpha =} \FloatTok{0.5}\NormalTok{, }
    \AttributeTok{outline =}\NormalTok{ outline.df) }\SpecialCharTok{+}
  \FunctionTok{geom\_text\_repel}\NormalTok{( }
    \CommentTok{\#family = "Sofia",}
                  \AttributeTok{min.segment.length =} \DecValTok{0}\NormalTok{) }\SpecialCharTok{+}
  \FunctionTok{annotate}\NormalTok{(}\StringTok{"point"}\NormalTok{, }\AttributeTok{x =}\NormalTok{ arielle[}\DecValTok{2}\NormalTok{], }\AttributeTok{y =}\NormalTok{ arielle[}\DecValTok{1}\NormalTok{], }\AttributeTok{color =} \StringTok{"red"}\NormalTok{) }\SpecialCharTok{+}
  \FunctionTok{scale\_fill\_viridis\_c}\NormalTok{(}\AttributeTok{direction =} \SpecialCharTok{{-}}\DecValTok{1}\NormalTok{) }\SpecialCharTok{+} 
  \FunctionTok{theme\_light}\NormalTok{() }\SpecialCharTok{+} 
  \CommentTok{\#theme(text = ggplot2::element\_text(family = "Sofia")) +}
  \FunctionTok{labs}\NormalTok{(}\AttributeTok{x =} \StringTok{"Longitude"}\NormalTok{,}
       \AttributeTok{y =} \StringTok{"Latitude"}\NormalTok{, }
       \AttributeTok{title =} \StringTok{"Which Grocery Store Should Arielle Go to?"}\NormalTok{) }\SpecialCharTok{+} 
  \ConstantTok{NULL}
\end{Highlighting}
\end{Shaded}

\includegraphics{denvergroceries_files/figure-latex/voronoi_map-1}

This gets me to the real platonic ideal of this analysis: no matter
where you are in the city, you can see where your closest grocery store
is and what its rating is. And we can see that Arielle's red dot really
is in the worst possible column and that she is relatively close to a
bodega with higher ratings and a 10 minute drive to some good Trader
Joe's locations. No need to ever go to the unsafeway again.



\end{document}
